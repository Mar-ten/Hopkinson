\documentclass[12pt]{article}
\usepackage{amsmath}
\usepackage{amsfonts}
\usepackage{mathrsfs}
\usepackage{lscape}
\usepackage{listings}
\usepackage{graphicx} % Allows for importing of figures
\usepackage{color} % Allows for fonts to be colored
\usepackage{comment} % Allows for comments to be made
\usepackage{accents} % Allows for accents to be made above and below text
%\usepackage{undertilde} % Allows for under tildes to take place for vectors and tensors
\usepackage[table]{xcolor}
\usepackage{array,ragged2e}
\usepackage{hyperref}
\usepackage{framed} % Allows boxes to encase equations and such
\usepackage{subcaption} % Allows for figures to be side-by-side
\usepackage{float} % Allows for images to not float in the document
\usepackage{booktabs}
%\usepackage[margin=0.75in]{geometry}
\usepackage[final]{pdfpages}
\usepackage{enumitem}
\usepackage[section]{placeins}

%%%%%%%%%%%%%%%%%%%%%%%%%  Function used to generate vectors and tensors %%%%%%%%%
\usepackage{stackengine}
\stackMath
\newcommand\tensor[2][1]{%
	\def\useanchorwidth{T}%
	\ifnum#1>1%
	\stackunder[0pt]{\tensor[\numexpr#1-1\relax]{#2}}{\scriptscriptstyle \sim}%
	\else%
	\stackunder[1pt]{#2}{\scriptscriptstyle \sim}%
	\fi%
}
%%%%%%%%%%%%%%%%%%%

\definecolor{mygrey}{rgb}{0.97,0.98,0.99}
\definecolor{codeblue}{rgb}{.2,0,1}
\definecolor{codered}{rgb}{1,0,0}
\definecolor{codegreen}{rgb}{0.3,0.33,0.12}
\definecolor{codegray}{rgb}{0.5,0.5,0.5}
\definecolor{codepurple}{rgb}{0.55,0.0,0.55}
\definecolor{codecyan}{rgb}{0.0,.4,.4}

\lstdefinestyle{mystyle}{
	backgroundcolor=\color{mygrey},   
	commentstyle=\color{codegreen},
	keywordstyle=\color{codeblue},
	stringstyle=\color{codepurple},
	numberstyle=\tiny\color{codegray},
	basicstyle=\footnotesize,
	breakatwhitespace=false,         
	breaklines=true,                 
	captionpos=b,                    
	keepspaces=true, 
	numbers=left,                    
	numbersep=5pt,                  
	showspaces=false,                
	showstringspaces=false,
	showtabs=false,                  
	tabsize=2
}
\lstset{style=mystyle}

\lstset{language=Matlab,backgroundcolor=\color{mygrey}}
\usepackage{lastpage}
\usepackage{fancyhdr}
\pagestyle{fancy}
%\lhead{\large{Nik Benko, John Callaway, Nick Dorsett, Martin Raming}} 
%\chead{\large{\textbf{ME EN 6960: Lab 1}}}
%\rhead{\today}
\cfoot{[\thepage\ of \pageref{LastPage}]}
\fancyheadoffset{.5cm}
\setlength{\parindent}{0cm}
\usepackage[left=.5in, right=0.50in, top=1.00in,bottom=1.00in]{geometry}
\usepackage{microtype} 
\usepackage{setspace}
\doublespace
%%%%%%%%%%%%%%%%%%%%%%%%%%%%%%%%%%%%%%%%%%%%%%%%%%%%%%%%%%%%%%%%%%%%%%%%%%
% git testing ii

\begin{document}
\title{ Determination of Dynamic Tensile Strength of Concrete Brazil Disc Specimens Using a Split Hopkinson Pressure Bar  \\ \normalsize{ME EN 6960}}
\author{Nik Benko, John Callaway, Nick Dorsett, Martin Raming}
\maketitle

% Nik
\begin{abstract} 

\end{abstract}

\section{Introduction} % Nik

\section{Methods}

\subsection{Experimental Techniques} 

\subsubsection {Split Hopkinson Pressure Bar} % Nick
%Gas gun, bar, location, etc
\subsubsection{High Strain Rate Data Acquisition} % Nick
%Strain gauges, scope

\subsubsection{Statistical Analysis} % John

\subsection{Procedure} % John

\subsection{Error and Uncertainties} % Nick 

\section{Results} % Martin

\section{Discussion} % Martin

\section{Conclusion} % Nik

% Bar speed and strain rate % Martin
% Forward and backward propogation of waves % Nick
% Conversion of voltage to strain, strain to force, force to strength % Nick
% Statistical Analysis % John

\section{Figures}
%\begin{figure}[H]
%	\centering
%	\includegraphics[width=1\textwidth]{DIC_Setup.png}
%	\caption{Experimental setup with close up of speckle pattern}
%	\label{fig:DIC}
%\end{figure}


\section{Tables}
%Recoded Experimental Data:
%\begin{table}[h]\footnotesize
%	\centering
%	\begin{tabular}{ |l|l|l|l| }
%		\hline
%		\multicolumn{2}{|c|}{\textbf{Mode I}}&\multicolumn{2}{|c|}{\textbf{Mixed Mode}}\\ \hline
%		\textbf{Load [N]} & \textbf{Image Number}&\textbf{Load [N]} & \textbf{Image Number}\\  \hline
%		0-5 & 7.784 & 0-4 & 25.58 \\ \hline
%		6& 7.784 & 5 & 27.58 \\ \hline
%		7 & 93.77 & 6 & 86 \\ \hline
%		8 & 215.6 & 7 &191.7 \\ \hline
%		9 & 295 & 8 & 324 \\ \hline
%		10 & 412 & 9 & 431 \\ \hline
%		11 & 489.6 & 10 & 486 \\ \hline
%		12 & 587 & 11 & 534.7 \\ \hline
%		13 & 745 & 12 & 629.1 \\ \hline
%		14 & 834 & 13 & 761.7 \\ \hline
%		15 & 899 & 14 & 805.3 \\ \hline
%		16 & 952 & 15 & 849.6 \\ \hline
%		17 & 1010 & 16 & 896 \\ \hline
%		-	& - & 17 & 1000 \\ \hline
%		
%		
%		
%	\end{tabular}
%	\caption{Loads and associated image number, first few images were used as reference images.}
%	\label{tab:data}
%\end{table}
%% All layups
%\
%\section{Appendix}
%
%%\subsection{Code}
%%
%%\begin{verbatim}
%%
%%\end{verbatim}
%
%\newpage
\bibliographystyle{IEEEtran}
\bibliography{Lab2Bib}
\end{document}